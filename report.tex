\documentclass[a4paper,12pt,twocolumn]{article}

\usepackage[utf8x]{inputenc}
\usepackage[english]{babel}
\usepackage[T1]{fontenc}
\usepackage{lmodern}

\usepackage{amsmath}
\usepackage{amssymb}
\usepackage{geometry}
\usepackage{a4wide}
\usepackage{enumerate}
\usepackage{graphicx}
\usepackage{lastpage}
\usepackage{cite}
\usepackage{listings}

\usepackage{lipsum}

\usepackage{fancyhdr}
\pagestyle{fancy}
\fancyhead{}
\fancyfoot{}

\lhead{Axel Angel}
\rhead{Scala BlitzView}
\chead{Report}
\rfoot{Page \thepage\ of \pageref{LastPage}}
\lfoot{\today}

\pdfinfo{
    /Author (Axel Angel)
    /Title (Scala BlitzView)
    /Subject (Report)
    }

% "define" Scala listing
\lstdefinelanguage{scala}{
  morekeywords={abstract,case,catch,class,def,%
    do,else,extends,false,final,finally,%
    for,if,implicit,import,match,mixin,%
    new,null,object,override,package,%
    private,protected,requires,return,sealed,%
    super,this,throw,trait,true,try,%
    type,val,var,while,with,yield},
  otherkeywords={=>,<-,<\%,<:,>:,\#,@},
  sensitive=true,
  morecomment=[l]{//},
  morecomment=[n]{/*}{*/},
  morestring=[b]",
  morestring=[b]',
  morestring=[b]"""
}
\lstset{language=scala}

\begin{document}

\begin{abstract}
% 1/2 lines of desc
Scala is a powerful language which currently provides a built-in library for non-strict views with some important shortcomings for the users such as unexpected and unintuitive behaviors.
In this work we created a new library, based on Scala Blitz, to provide lightweight, non-strict and parallel-efficient collections, we call them Views.
We present the library API design, implementation and how programmers can use and extend it.
\end{abstract}

\section{Introduction}
% subject of the work
% scala (powerful, fast-moving lang
% standard collections, non-strict and views
Scala is a powerful and fast-moving language that fuses object-oriented programming with a wide range of function programming concepts \cite{scala-overview}.
It runs on the JVM and as such it stays compatible with Java and its ecosystem.
Scala itself provides an important number of libraries, for example Scala collection, which implements Lists, Arrays, Maps and Sets with immutable and mutable variants.
They are more in accord within the Scala environment than the Java collections, moreover they provide the functional programming concepts like constructors.

A {\it View} in Scala is a non-strict version of some collection set.
{\it Non-strictness} here is a mean to post-pone computations over a collection until the final result is actually needed, this type of view is called a proxy.
The View is said to be {\it forced} when the computation need to be performed over all the elements.
A View captures the operations that are postponed over its inner collection in constant memory $O(1)$ and stacks them to provide efficient computation in a single pass over the collection $O(n)$.

In practice this is used when multiple operations, such as multiple \verb|map| and \verb|filter|, are called consecutively.
As Views are usually {\it immutable}, as in our design, performing a new operation actually returns a new View where all previous operations are captured along the new one.
Immutability greatly simplifies the implementation and open new possibilities for the programmer to combine and reuse Views in his code.

Therefor a View allows us to use special optimisations such as merging these operations to compute them all at once for each element of the inner collection.
As the operations are done element by element, we can split the inner collections into a dynamic number of chunks and compute the operations in parallel depending on the number of cores of the computer.

The design of the Views API is primordial because it can greatly limit the optimisations thus influencing the efficiency of the computations, as far as deciding whether they can be done in parallel and not.
There exists two types of operations over Views:
\begin{description}
    \item[Transformers:] These can be postponed and captured in the View without evaluating (forcing) the elements, eg: \verb|map| and \verb|filter|. Usually their type is \verb|[a] -> [b]|.
    \item[Folders:] These are the last operations that actually force the View to be computed and in general returns a single element, eg: \verb|aggregate| and \verb|max|. Usually their type is \verb|[a] -> b|.
\end{description}
In this work, we focused on a powerful subset of the actual Scala collections API to preserve the efficiency of the Views while providing very powerful and functional non-strict collections.

\section{Previous works}
Scala and its collection offer a large toolbox of functions taken from functional paradigm such as \verb|flatMap| and \verb|aggregate| in a object-oriented hierarchy of classes with common interfaces.
This collection interface is declared in the parent class \verb|Traversable|\cite{scala-collections} which is inherited by multiple types of collection in order to provide a common API that operates uniformly on all these different structures transparently for the programmer: whatever he uses is a \verb|List|, an \verb|Array|, a \verb|Map|, a \verb|LinkedList| or any descent of these classes, they all share this common methods.
The programmer has to learn and understand it once, then he can use his experience for any of these collections easily: it's intuitive and greatly increase the productivity.
The built-in collections in Scala are strict in the sense all operations are directly computed because Scala is a strict language, although the programmer can specify the \verb|lazy| variable-keyword, this doesn't solve the problem optimally.

% scala views, broken
Since Scala 2.8, the Views joined the built-in toolbox to offer non-strict collections using the common interface of collections.
They allow to create a proxy over a collection that captures the operations on them until an operation force it.
The purpose of the proxy is to change the evaluation strictness of the collections by handling the computation itself when it see fits.
For example a call to \verb|flatMap| over a View returns immediately whereas over a strict collection this may take some time to return.
This is done by implementing all methods of the collection interface in a way the operations are remembered and done when necessary
The wrapper is kept to use non-strict operations then the programmer can force the conversion into a regular collection: this is done by unwrapping the proxy after the computations and filling a regular collection with the result.
This design decision has great advantages when it comes to people experienced with Scala collections because there is no external difference between them.
Unfortunately it has two important costs for Scala in terms of the implementation and for the programmer who expect consistent and efficiently results.
We will develop these aspects in the section and how we approached differently the problem.

Independently, Scala added later a way to convert collections to parallel variants in order to compute the operations with multiple cores.
The thin wrapper is specialized depending on the underlying type, most of the types requires constant time for this conversion.
The wrapper provides the same interface with the usual collections, thus there is no difference again in the code after the conversion.
The programmer applies the methods as usual then he can call a method to convert back to the regular collections (unwrapping).

% talk about java 8 streams, limitations/bad design? (adv: unboxed?)
% (java: problem type dependent)
Java 8 was recently released with a new toolbox dedicated to functional programming (eg: lambda functions) and the new package Stream.
These concepts allow the programmer to finally manipulate concisely sequences with \verb|flatMap| and the similar functions well known in Scala.
The concept of Stream in Java 8 is different of the Streams in Scala.
The first is the Java implementation of the non-strict Views we discussed above whereas Scala Streams are infinite non-strict sequences, usually defined recursively.
Moreover the Java Stream can be converted to a parallel variant as the Scala Views, the main difference is that Java implemented a specialized version only for Streams and the interface is very different than the usual Java collections.
Java Stream and Scala Views have an important number of common methods such as \verb|flatMap|, \verb|find|, \verb|min|/\verb|max| and the like.
They both wrap the inner structure and require the programmer to call specific methods to unwrap, such as \verb|toArray|, or when he calls a folding method.
They both require to explicitly convert to non-strict versions and then to parallelized variants if needed.
The main difference is that Java has severe limitations with return values depending on Generics type: this is visible for all variants of \verb|flatMap| whose name is postfixed by the type explicitly, eg: \verb|flatMapToInt| and return a specialized type, eg: \verb|IntStream|.
An important problem with Java Stream is the lack of transparent referenceability and reusability: after a terminal operation (Folders) on a Stream, it cannot be reused, it is consumed by the operation (side-effect) and can never be reused.
This limits greatly the combinatorial power of Streams as one needs to create new Stream for each new use whereas a single View can be reused alone and be part of other Views.

\section{Views}
We now define the properties of Views and describe the constraints we must satisfy in our API based on the experience of the previous works.

% what operations/methods make sense?
As we said, Views are non-strict collections and they guarantee {\it constant} time and {\it constant} memory for transformers.
This is possible because the View (the proxy that wraps the underlying collection) remembers all transformers the programmer requested.
As the computations are bookmarked into the view internals, no change is actually made to the inner collection, these only happen in the proxy.
In the current Scala infrastructures, it was decided these Views are immutable, thus each time a transformer is applied on it, a new View is returned.
Multiple advantages are offered this way: first the programmer can rely on the immutability, for example he can store multiple Views over the same data without worrying about side-effects on his original collection nor his intermediate Views.

Views should be seen as an adaptor over a collection where each element passes through its pipeline made of operations (the transformers) which are computed and are collected by the last operators (the folder) as they pass by.
The choice of whether an operation is a transformer or a folder will depend on the internal implementation of the library.

% why scala views broken, which methods useless (never finish)
The problem that interest us in this project was to overcome the limitations seen in the current implementation of Views.
One such problem is due to the fact Views inherit from the whole collection API which contains all usual operations that were designed to work on strict and mostly on sequential structures.
% permutations
Although operations such as \verb|permutations| makes perfectly sense for the usual collections, these operations cannot be efficiently implemented in the case of Views.
Despite this fact, this kind of operations is available in current Views but they are not correctly implemented which can make it crash, leading to an unexpected behavior:

\begin{lstlisting}
val xs = 0 to 3
xs.permutations.toArray
// correct result
xs.view.permutations.toArray
// UnsupportedOperationException
\end{lstlisting}

There exists other operations that doesn't play nice when they are used on Views and this is a problem for programmer who expect the least surprise.
% flatten -> list
For example \verb|flatten| does not return a flatten View but a new List containing the result of the flattening, even if we use Views inside and outside, this operation should have been non-strict as well.

Other important problems with Scala Views arise when we want to combine non-strictness of Views with parallelism of Par together.
The following is permitted although one version doesn't make sense:

\begin{lstlisting}
val xs = (0 to 1000).par.view
val ys = (0 to 1000).view.par
\end{lstlisting}

Which one is correct? Are they equivalent? In fact they are not equivalent, worse, the second version seem to loose its non-strictness.

From these problems we can see there is a leak of coordination between the collection API and the way we can construct parallelized Views.
Scala collections offer too many methods that cannot be efficiently implemented or that does not make sense in a non-strict context, and the use of both Views and Par together should be done in a unified way to avoid these problems.

% use inherited collections API -> broken

\section{Design}
In this work we propose an alternative implementation of Scala Views that solves the issue of coordination between available methods and efficiency in non-strict and parallelized context.

% problem of inefficiency, why cannot implement certain?
The first design decision we made is to create a new interface, a trait, that does not contain problematic methods.
There are different types of such methods: some are inherently sequential (eg: \verb|reduceRight|), some require forcing the View (eg: \verb|ordered|), some are inefficient anyway (eg: \verb|permutations|) and some are possible but trickier to implement (eg: \verb|takeWhile|).

We focused our prototype on the most important ones:
\begin{itemize}
    \item \verb|[a] -> [b]|: \verb|map|, \verb|filter| which are transformers.
    \item \verb|[a] -> b|: \verb|aggregate| is the most important. It is the building block for the other folders such as \verb|min|, \verb|sum|, \verb|find|, \verb|exists|, \verb|count| which are folders.
\end{itemize}

% hierarchy: interface trait, impl-spec trait, classes
The transformers are represented by the trait \verb|ViewTransform[-A, +B]|, in our internal implementation.
It represents a function from $A$ to $B$ where $A$ is contravariant and $B$ covariant.
This trait is used to pipeline operations when we are folding: we first apply our transformations then we fold over a given function, this is the purpose of the method \verb|fold|.
The important feature of these transformers are they are stackable: a transformer can contain an other transformer and so on, this is the purpose of \verb|>>|.
In our design, there are three types of transformers: \verb|Map| which applies a function on each element, \verb|Filter| which drops elements according to the given predicate function and \verb|Identity| used at the bottom of the stack.

The second design decision is the structure of the class hierarchy for our Views:
\begin{description}
    \item[{\tt BlitzView}]: the top trait that describe the available operations (transformers and folders) on all Views.
        It contains all the methods we just discussed above.
    \item[{\tt BlitzViewImpl}]: contains the trait for our implementation.
        Anyone is free to create a new implementation next to it, see section \ref{sec:ext}.
        This trait inherits \verb|BlitzView| and provides the common implementation of all methods for subclasses in terms of a method \verb|aggInternal|.
        The children classes then must only implement this method to inherit all operations of this design.
    \item[{\tt BlitzViewC}]: is the View that contains an underlying collection.
        This is the class that is used as a proxy closest to a wrapped collection and the only one that actually captures operations in a stack.
    \item[{\tt BlitzViewVV}]: is the View that concatenates two Views together.
        This would be the result of \verb|++| on two Views.
    \item[{\tt BlitzViewFlattenVs}]: is the View that contains a list of View and flatten them together in a single View.
\end{description}

% implicit extension: ext-method, evidence

\section{Usability}
% external trait, interface

Let's take an example to illustrate how one can use Views:

\begin{lstlisting}
val xs = (0 to 10).toArray
val v = xs.bview
val u = v.map(_ + 10)
\end{lstlisting}

The programmer has the guarantee \verb|xs| will never be affected by the actions he is performing on the proxies, here \verb|v| or \verb|u|.
Moreover \verb|u| is independent of \verb|u| and both Views can be used as many time as needed.

The very nice properties of Views are important because they increase the possible use cases.
For example Scala Views support mutable collections: by creating a View on an underlying mutable HashMap, the programmer can create a view over his data as on can do in SQL where as the underlying data change views are always up to date accordingly.

\begin{lstlisting}
import collection.mutable.HashMap
val m = HashMap((1,2))
val v = m.view.map(case (x,y) => x+y)
v.toArray // Array(3)
v.sum // 3
m += ((3,4))
v.toArray // Array(7, 3)
v.sum // 10
\end{lstlisting}

\section{Extensibility}
\label{sec:ext}
% internal trait, how to extend, aggInternal, BlitzViewImpl
\lipsum[6]

\section{Conclusion}
% TODO: provide source code (tagged commit)
% future work, what's next?
\lipsum[7]

\bibliography{report}{}
\bibliographystyle{plain}


\end{document}
